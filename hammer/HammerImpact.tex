\documentclass[11pt,a4paper]{article}

\title{Hammer Hit}
\author{Peter Borenstein}
\date{2018 July}

\begin{document}
\maketitle
A hammer works by storing energy which is transferred to the target in a collision.
The target could be a nail or demolitioning concrete.
Three ways of viewing the energy of a hammer come to mind;
the hammer has speed when the target is hit,
you put work into the hammer when swinging,
and you put work into the hammer when raising it over your head.

Viewing the hammer from different perspectives should come to the same conclusion on how hard the hammer hits.
Using different equations should give the same result.
Energy is not created or destroyed, so all the effort you put into the hammer will exist at the collision.

To keep the math simple, let's assume the hammer weighs \textbf{5 kilograms} (11lbs), we lift the hammer \textbf{3 meters} up, the hammer pauses at its peak, and our muscles consistently push down on the hammer with \textbf{100 Newtons} (the force required to hold up 22 lbs).
These three factors give much to ponder.
The math gets complicated when you consider technique and swinging in an arch, so lets not go into that\ldots

To understand and to be understood, we need to use standard units for common equations.
Table \ref{tab:units} gives units and abbreviations used.

\begin{table}
\noindent\begin{tabular}{l | c | l | c}
Thing&symbol&unit&abbreviation\\
\hline
force&F&newtons&$N$\\
distance&d&meters&$m$\\
height&h&meters&$m$\\
mass (weight)&m&kilograms&$kg$\\
time&t&seconds&$s$\\
energy&E&joules&$J$\\
speed&v&meters per second&$m/s$\\
acceleration&a&meters per second squared&$m/s^2$\\
gravity of Earth&g&meters per second squared&$m/s^2$\\
\end{tabular}
\label{tab:units}
\caption{Units}
\end{table}
\pagebreak

First, let's consider the total effort you put into pushing a hammer down.
The work you do is the force applied multiplied by distance traveled. $$Work = F*d$$
Interestingly, this does not consider the weight of the hammer! We have: $$100N*3m = \mathbf{300J}$$

Second, let's consider the speed the hammer is traveling when it hits the target. The energy of a moving object is half the mass multiplied by the speed squared. $$E_k= 1/2mv^2$$

We must determine the hammer's speed at impact, $v$.
Motion and position equations are so common that they are given a special name, the Kinematic Equations. The one we need here is
$$v=at+vi$$
In our situation, the hammer is breifly still before it is pushed down. The initial speed, $vi$, is zero; ($v = at + 0$).
The speed increases by acceleration over time.

The time, $t$, is the difference between when we start pushing the hammer down and when the hammer hits.
We need another Kinematic Equation.
$$d = vi*t + 1/2at^2$$
The distance, $d$, traveled is 3m. Rearranged to find time without a $vi$,
$$t = \sqrt{2*\frac{d}{a}}$$

We must determine the acceleration, $a$. Issac Newton told us $F=ma$. Rearranged, acceleration is determined by our force divided by the hammer's weight:
$$a=F/m$$
\pagebreak

Finally, with no more mystery letters, we can solve the chain of equations based on the original three conditions.
$$a = 100N/5kg = 20m/s^2$$
$$t=\sqrt{2*\frac{3m}{20m/s^2}}= 0.55s$$
$$v = 0.55s * 20m/s^2= 11m/s$$
$$E_k=\frac{1}{2}*5kg*(11m/s)^2\approx\mathbf{300J}$$

Voila! Considering speed and work both result in the hammer hitting with \textbf{300J} of energy.
If you punched the second equation into your calculator, you would actually get 302J because the previous numbers are rounded.
\pagebreak

So far, we have neglected gravity. Gravity pulls all objects down at about $9.8m/s^2$.
Air resistance will slow down light objects, but not the hammer.

The work we did lifting up our hammer gives it potential energy.
Energy in motion is kenetic. These energies are represented $E_p$ and $E_p$ respectively.
Potiential energy would be see if we let the hammer fall without pushing it at all.

Potential energy is given by the equation: $$E_p=mgh$$
In our situation, we have $E_p=5kg*9.8m/s^2*3m=\mathbf{147J}$

To prove we will get this same energy from speed by letting go of the hammer, we can solve the speed equations again with only gravity:
$$a=g=9.8m/s^2$$
$$t=\sqrt{2*\frac{3m}{9.8m/s^2}}= 0.782s$$
$$v = 0.782s * 9.8m/s^2= 7.7m/s$$
$$E_k=\frac{1}{2}*5kg*(7.7m/s)^2\approx\mathbf{147J}$$

We can verify the energy a third time by calculating the work done by gravity:
$$F=ma=5kg*9.8m/s^2 = 49N$$
$$Work = F*d = 49N*3m = \mathbf{147J}$$

We get the same answer from considering the energy in a hammer's impact from 3 perspectives.

Our work and the work from gravity adds together because we are pushing down. Forces in the same direction add togther.
The numbers should show the total energy is $300J + 147J = \mathbf{447J}$.
$$F =  100N + 49N = 149N$$
$$a = F/m = 149N/5kg = 29.8m/s^2$$
$$t=\sqrt{2*\frac{3m}{29.8m/s^2}}= 0.45s$$
$$v = 0.45s * 29.8m/s^2= 13.4m/s$$
$$E_k=\frac{1}{2}*5kg*(13.4m/s)^2\approx\mathbf{447J}$$

Knowing why the hammer works does not make swinging it any easier...

\end{document}

